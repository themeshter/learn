\documentclass[12pt,letter]{memoir} 

%%%% Begin latex configuration

% line spacing
%\linespread{1.3}

% configuration for the bibtex package
% redefine the info from /usr/share/texmf/tex/latex/biblatex/lbx/english.lbx
\usepackage{booktabs}
%\usepackage[style=alphabetic, hyperref=true]{biblatex}
\usepackage[style=numeric, hyperref=true]{biblatex}
\usepackage{supertabular}
\bibliography{document}

% command for code
\DeclareFixedFont {\codefont}{OT1}{pcr}{m}{n}{10}
\newcommand{\code}[1]{\codefont#1\normalfont}
% command for reader's question
\newcommand{\readq}[1]{\ding{44}~\textbf{Maybe you would say: }"#1"}
\newcommand{\question}[1]{\ding{45}~\textbf{Try to answer: }"#1"}
\newcommand{\newcodeline}{\ding{229}\\}
\newcommand{\checkbox}{\SquareShadowBottomRight}

% index
\usepackage{makeidx}
%\usepackage{showidx}
%\makeindex[document] removed from here

% shape packages
\usepackage{bbding}
\usepackage{pifont}
% Warning: wasysym has a conflict with glossaries
%\usepackage{wasysym}
%\usepackage{dingbat}

% notes
\usepackage{fancybox}
\newenvironment{cbox}%
  {\begin{center}\begin{Sbox}\begin{minipage}{0.8\textwidth}}%
  {\end{minipage}\end{Sbox}\shadowbox{\TheSbox}\end{center}}

% the hyperref config
\usepackage{hyperref}
\hypersetup{colorlinks=true}
\usepackage{color,calc,graphicx,soul,fourier}
% the listings settings
\usepackage{listings}
\definecolor{listingbackground}{gray}{0.95}
\lstset{frameround=tttt}
\lstset{frame=tRBl}
\lstset{backgroundcolor=\color{listingbackground}}
\lstset{captionpos=b}
\lstset{basicstyle=\footnotesize}
\lstset{language=[LaTeX]TeX}

% glossaries
\usepackage[toc,acronym,automake]{glossaries-extra}
\makeglossaries
%\include{extra/glossary}

% chapter style
\renewcommand{\chaptername}{Problem}
\definecolor{chaptercolor}{rgb}{0,0,0}
\makeatletter
\newlength\dlf@normtxtw
\setlength\dlf@normtxtw{\textwidth}
\def\myhelvetfont{\def\sfdefault{mdput}}
\newsavebox{\feline@chapter}
\newcommand\feline@chapter@marker[1][4cm]{%
  \sbox\feline@chapter{%
     \resizebox{!}{#1}{\fboxsep=1pt%
       \colorbox{chaptercolor}{\color{white}\bfseries\sffamily\thechapter}%
     }}%
  \rotatebox{90}{%
     \resizebox{%
       \heightof{\usebox{\feline@chapter}}+\depthof{\usebox{\feline@chapter}}}%
     {!}{\scshape\so\@chapapp}}\quad%
  \raisebox{\depthof{\usebox{\feline@chapter}}}{\usebox{\feline@chapter}}%
}
\newcommand\feline@chm[1][4cm]{%
  \sbox\feline@chapter{\feline@chapter@marker[#1]}%
  \makebox[0pt][l]{% aka \rlap[\Checkmark]
     \makebox[1cm][r]{\usebox\feline@chapter}%
  }}
  
\makeindex[document]
\makechapterstyle{daleif1}{
  \renewcommand\chaptitlefont{\normalfont\Huge\bfseries\scshape\color{chaptercolor}}
  \renewcommand\chapternamenum{}
  \renewcommand\printchaptername{}
  \renewcommand\printchapternum{\null\hfill\feline@chm[2.5cm]\par}
  \renewcommand\afterchapternum{\par\vskip\midchapskip}
  \renewcommand\printchaptertitle[1]{\chaptitlefont\raggedleft ##1\par}
}
\makeatother
\chapterstyle{daleif1}
\setlrmarginsandblock{25mm}{20mm}{*}
\setulmarginsandblock{20mm}{20mm}{*}
\setsecnumdepth{subsection}
\maxsecnumdepth{subsection}
\maxtocdepth{subsection}
\settocdepth{subsection}

% PDF nice output
%\hypersetup{colorlinks=true, citecolor=black}
% configuration for the page size
%\setstocksize{297mm}{210mm}
%\settrimmedsize{297mm}{210mm}{*}
%\setlength{\trimtop}{0pt}
%\setlength{\trimedge}{\stockwidth}
%\addtolength{\trimedge}{-\paperwidth}
%\settypeblocksize{252mm}{175mm}{*}
%\setulmargins{20mm}{*}{*}
%\setlrmargins{25mm}{*}{*}
%setul and lr marginblock commands removed from here
%\setmarginnotes{17pt}{51pt}{\onelineskip}
%\setheadfoot{\onelineskip}{2\onelineskip}
%\setheaderspaces{*}{2\onelineskip}{*}
\checkandfixthelayout


% some custom-made commands
\newcommand{\picwidth}{.75\textwidth}
\newcommand{\cstdfigure}[2]{
\begin{figure}[!ht]
 \centering
 \includegraphics[width=\picwidth]{pics/#1}
\caption{#2} \label{#1}
\end{figure}
}
\newcommand{\cstdfigurewpercentage}[3]{
\begin{figure}[!ht]
 \centering
 \includegraphics[width=#3\textwidth]{pics/#1}
\caption{#2} \label{#1}
\end{figure}
}
\newcommand{\carrow}{~\ding{213}~}
\newcommand{\biarrow}{~\ding{214}~}

\newcommand{\eg}{e.g.~}
\newcommand{\fastCenter}[1]{
\begin{center}
\Large{#1}
\end{center}
}

\newcommand{\titlepage}[1]{
\fastCenter{Introduction to Mathematical Thinking\\Course by Stanford University}
\vspace{\fill}
\begin{center}
\HUGE{#1}
\end{center}
\vspace{\fill}
\fastCenter{Coursera\\\today}}

% changes in the font family
\usepackage{bookman}
\usepackage{helvet}
\renewcommand{\rmdefault}{ppl}
\renewcommand{\sfdefault}{phv}

%hypenation
% document numbering
%\setsecnumdepth{subsection}
%\maxsecnumdepth{subsection}
%\maxtocdepth{subsection}
%\settocdepth{subsection}
%above moved to conditional

%%%% End latex configuration

% mainmatter
\begin{document}
%general settings
\tightlists

\frontmatter
\pagestyle{empty}

% title pages
\titlepage{\textbf{Peer-graded Assignment}}
\cleardoublepage

\pagenumbering{roman}
\pagestyle{headings}
\tableofcontents
%\clearpage
%\listoffigures
%\clearpage
%\lstlistoflistings
%\clearpage
\newpage
%\listoftables
\mainmatter
\pagenumbering{arabic}
%\chapter*{Abstract}
%\include{authors/authors}

\chapter*{Introduction}
This material has been developed as a solution to the peer-graded exercise for the September-November session of \href{https://www.coursera.org/learn/mathematical-thinking}{``Introduction to Mathematical Thinking''} by \href{http://profkeithdevlin.com/}{Prof. Keith Devlin} course available on \href{http://www.coursera.com}{Coursera}.

During the peer-graded review I received some supportive feedback (the grade was 93\%) and some of the fellow students asked about how the material was created. Therefore I decided to publish the work, as an example for further use.

Note that the reader of this material must not use nor present the content of this material as being their own work (in other words, you must obey \href{https://learner.coursera.help/hc/en-us/articles/209818863-Coursera-Honor-Code}{Coursera's honor code}). The \emph{content} (the proofs themselves) should be used just as an inspiration on how to do your own work (beware that I am not a mathematician, so my approach in some cases might not be the correct one - so use at your own risk!).

However, the \emph{template} (the \LaTeX~file) can be used even for commercial purpose being released under the \href{https://github.com/themeshter/learn/blob/master/LICENSE}{MIT license}.

\section{Editorial Notes}
This material was created using the following tools:
\begin{itemize}
\item \textbf{Operating system}: \href{http://releases.ubuntu.com/16.04/}{Ubuntu 16.04.3 LTS (Xenial Xerus) x64 Desktop Edition}.
\item \textbf{\LaTeX~editor}: Texmaker version 4.4.1.
\item \textbf{\LaTeX~toolset}: Texlive.
\end{itemize}
There are no links provided to Texmaker or Texlive because you can install those from Ubuntu's distribution.

In order to start your work, follow these steps:
\begin{enumerate}
\item Clone this repository.
\item Open the \code{main.tex} file with Texmaker.
\item Start editing it and change it to whatever you want. The \code{F1} key will ``build'' your PDF.
\end{enumerate}

If you run into any trouble with producing your PDF:
\begin{itemize}
\item Get help from yourself ;-). By saying that I mean just look online help (``google it'', colloquially speaking) before asking somebody else. You will be surprised how many problems you can actually solve by yourself with no external help at all.
\item If the step above doesn't work, call a friend - somebody you know that has experience with \LaTeX. I am not a \LaTeX expert, but with online help I usually get to where I want to go...
\item As a last resort, you can open an issues here and I will try to respond when I will have some time (answering is not guaranteed though).
\end{itemize}

As a last remark, if you find that this was helpful to you, pass it or or even star it on github!

\chapter{Problem 1}
\section*{Statement}
Say whether the following is true or false and support your answer by a proof.
\begin{equation}\label{prob1}
(\exists m \in \mathbb{N})(\exists n \in \mathbb{N})(3m+5n=12)
\end{equation}
\section*{Solution}
We will demonstrate that it is not possible to find two natural numbers (integer numbers strictly greater than $0$) that satisfy the equation $3m+5n=12$.

Given that 12 is an even number, it means that both $3m$ and $5n$ must be simultaneously either odd or even. Because 3 and 5 are odd, it means that both $m$ and $n$ must be simultaneously odd or even. Let's start with having $m$ and $n$ odd.
The smallest numbers we can consider are $m=1$ and $n=1$ which means that
\begin{equation}\label{both_odd}
\begin{split}
3m+5n & = \\
 & = 3\cdot1+5\cdot1 \\
 & = 8
\end{split}
\end{equation}
With this combination, the equality we aim form is not satisfied (see equation \ref{both_odd}). The next feasible possibility would be to consider $m$ and $n$ both even and the smallest value for that would be 2. In that case,
\begin{equation}\label{both_even}
\begin{split}
3m+5n & = \\
 & = 3\cdot2+5\cdot2 \\
 & = 16
\end{split}
\end{equation}
Equation \ref{both_even} not only shows that the result is not equal to the desired one (12) but it is also greater. Increasing $m$ or $n$ will only increase the value of $3m+5n$, so there is no possible combination of $m$ and $n$ to satisfy equation \ref{prob1}, hence the statement is false. $\square$
\chapter{Problem 2}
\section*{Statement}
Say whether the following is true or false and support your answer by a proof: The sum of any five consecutive integers is divisible by 5 (without remainder).
\section*{Solution}
Let's consider an arbitrary integer number $n$ and the four consecutive numbers following it: $n+1$, $n+2$, $n+3$, $n+4$. The sum of these five numbers can me expressed as:
\begin{equation}\label{fiveconsecutive}
\begin{split}
n+(n+1)+(n+2)+(n+3)+(n+4) & = \\
 & =  5n+10\\
& = 5(n+2)
\end{split}
\end{equation}
Given that $5\mid5(n+2)$ we can conclude that:
\begin{equation}
(\forall n \in \mathbb{Z})(5 \mid \sum_{i=0}^{4}(n+i))
\end{equation}
$\square$
\chapter{Problem 3}
\section*{Statement}
Say whether the following is true or false and support your answer by a proof: For any integer $n$, the number $n^2 + n + 1$ is odd.
\section*{Solution}
In order to prove that this statement is true we will leverage the following rules for integer numbers addition (their proof is not covered here):
\begin{enumerate}
\item The sum of two integers with the same parity (either both odd or both even) is an even integer.
\item The sum of two integers having different parities (one being odd and the other being even) is an odd integer.
\end{enumerate}
We can express $n^2 + n + 1$ as $n(n+1) + 1$.
For every $n$ we would choose, $n(n+1)$ will be an even number because at least one of $n$ or $n+1$ is even. Based on rule number 2 expressed above, $n(n+1)+1$ will be an odd number because it represents the sum between an even number ($n(n+1)$) and an odd number ($1$).$\square$

\chapter{Problem 4}
\section*{Statement}
Prove that every odd natural number is of one of the forms $4n + 1$ or $4n + 3$, where $n$ is an integer.
\section*{Solution}
If $k$ is an odd natural number then $(\exists p \in \mathbb{Z})(k=2p+1)$. At its end, $p$ can be either odd or even, so we have two scenarios:
\begin{enumerate}
\item If $p$ is even, then $(\exists n \in \mathbb{Z})(p=2n)$ and subsequently $k=2\cdot2n+1=4n+1$.
\item If $p$ is odd, then $(\exists n \in \mathbb{Z})(p=2n+1)$ and subsequently $k=2\cdot(2n+1)+1=4n+3$.
\end{enumerate}
These two scenarios prove that every odd natural number is of one of the forms $4n + 1$ or $4n + 3$, where $n$ is an integer.$\square$

\chapter{Problem 5}
\section*{Statement}
Prove that for any integer $n$, at least one of the integers $n$, $n + 2$, $n + 4$ is divisible by 3.
\section*{Solution}
An integer $n$ can be one of $3k$, $3k+1$, $3k+2$, where $k$ is another integer. Formally,
\begin{equation}\label{3k}
(\forall n \in \mathbb{Z})(\exists k \in \mathbb{Z})((n=3k) \lor (n=3k+1) \lor (n=3k+2))
\end{equation}
Let's analyze the three situations one by one:
\begin{enumerate}
\item $n=3k$. In this situation $3 \mid n$ because $3 \mid 3k$.
\item $n=3k+1$. In this case, because $n+2=3k+1+2=3k+3=3(k+1)$ then $3 \mid (n+2)$.
\item $n=3k+2$. In this case, because $n+4=3k+2+4=3k+6=3(k+2)$ then $3 \mid (n+4)$.
\end{enumerate}
As a result, for any chosen $n$, at least one of $n$, $n + 2$, $n + 4$ is divisible by 3.$\square$

\chapter{Problem 6}
\section*{Statement}
A classic unsolved problem in number theory asks if there are infinitely many pairs of `twin primes', pairs of primes separated by 2, such as 3 and 5, 11 and 13, or 71 and 73. Prove that the only prime triple (i.e. three primes, each 2 from the next) is 3, 5, 7.
\section*{Solution}
In problem 5, we have proven the following statement: for any integer $n$, at least one of the integers $n$, $n + 2$, $n + 4$ is divisible by 3. We will leverage this result to solve the current problem.

We will use the method of proving by contradiction. Let's assume that there is a prime number $n>3$ so that $n$, $n+2$ and $n+4$ are prime. Based on the proof from problem 5, we know that at least on of the three mentioned numbers are divisible by 3. This means that:
\begin{enumerate}
\item Either the primes are 3,5,7 (which cannot be because we chose $n>3$) or
\item At least one of the numbers $n$, $n+2$ and $n+4$ is not prime since it is divisible by 3.
\end{enumerate}
This contradicts our original assumption, so there is no $n>3$ for which $n$, $n+2$ and $n+4$ are all prime.$\square$

\chapter{Problem 7}
\section*{Statement}
Prove that for any natural number $n$,
\begin{equation}
2 + 2^2 + 2^3 + \cdots + 2^n = 2^{n+1}-2
\end{equation}
\section*{Solution}
To prove this, we will use the method of mathematical induction.

As a fist step, we will validate that the equation is valid for $n=1$: $2^1=2^{1+1}-2$ is a true statement since $2=4-2$.

Applying the induction step, let's assume that the following equation is true for $n$:
\begin{equation}
2 + 2^2 + 2^3 + \cdots + 2^n = 2^{n+1}-2
\end{equation}
... and let's prove that it holds true for $n+1$:
\begin{equation}\label{inducepown}
\begin{split}
2 + 2^2 + 2^3 + \cdots + 2^n + 2^{n+1} & = \\
(2 + 2^2 + 2^3 + \cdots + 2^n) + 2^{n+1} & = \\
2^{n+1}-2+2^{n+1} & = \\
2 \cdot 2^{n+1}-2 & = \\
2^{n+1+1}-2 & = \\
& = 2^{n+2}-2
\end{split}
\end{equation}

Equation \ref{inducepown} proves that the induction step holds true for $n+1$, thus completing the proof.$\square$

\chapter{Problem 8}
\section*{Statement}
Prove (from the definition of a limit of a sequence) that if the sequence $\{a_{n}\}_{n=1}^{\infty}$ tends to limit $L$ as $n\to\infty$, then for any fixed number $M>0$, the sequence $\{Ma_{n}\}_{n=1}^{\infty}$ tends to the limit $ML$.
\section*{Solution}
By the definition of a limit for sequence $\{a_n\}_{n=1}^{\infty}$, we have:
\begin{equation}\label{limitofequation}
(\forall \epsilon>0)(\exists N \in \mathbb{N})(\forall n \in \mathbb{N})(n \geq N \implies |a_n-L|<\epsilon)
\end{equation}
With the definition expressed in \ref{limitofequation}, given an $\epsilon$ value, we will pick an integer number $N$ (let's call it $N_M$) having a sufficiently high value to satisfy the following inequality:
\begin{equation}\label{eplsioninequality}
|a_n-L|<\frac{\epsilon}{M}
\end{equation}
If we multiply the inequality \ref{eplsioninequality} with $M$ (we can do this because $M$ is positive), then we can state that:
\begin{equation}\label{Meplsioninequality}
|Ma_n-ML|<\epsilon
\end{equation}
This way we can restate the definition of a limit for a sequence as this:
\begin{equation}\label{limitofequation2}
(\forall \epsilon>0)(\exists N_M \in \mathbb{N})(\forall n \in \mathbb{N})(n \geq N \implies |Ma_n-ML|<\epsilon)
\end{equation}
Applying the mentioned definition for the sequence $\{Ma_{n}\}_{n=1}^{\infty}$, we can conclude that its limit is $ML$.$\square$

\chapter{Problem 9}
\section*{Statement}
Given an infinite collection $A_n$, $n=1,2,\dots$ of intervals of the real line, their \emph{intersection} is defined to be
\begin{equation}\label{intersection}
\bigcap\limits_{n=1}^{\infty}A_n=\{x \mid (\forall n)(x \in A_n)\}
\end{equation}
Give an example of a family of intervals $A_n$, $n=1,2,\dots,$ such that $A_{n+1} \subset A_n$ for all $n$ and $\bigcap\limits_{n=1}^{\infty}A_n=\emptyset$. Prove that your example has the stated property.
\section*{Solution}
Let us assume the following collection of intervals: $A_n=(0, \frac{1}{n})$. To exemplify, our intervals are:
\begin{equation}\label{anseq}
\begin{split}
A_1 = (0, 1) \\
A_2 = (0, \frac{1}{2}) \\
A_3 = (0, \frac{1}{3})\\
\cdots \\
A_n = (0, \frac{1}{n})\\
\end{split}
\end{equation}
First we will prove the inclusion and then we will show that there is no common element between all these intervals.
\subsection*{Proof of Inclusion}
For intervals $A_n$ and $A_{n+1}$, we can state that:\noprelistbreak
\begin{enumerate}
\item they have the same lower bound, $0$.
\item because $\frac{1}{n+1}<\frac{1}{n}$, the upper bound of $A_{n+1}$ is lower that the upper bound of $A_n$.
\end{enumerate}
Given these two statements, it means that $A_{n+1} \subset A_n$.
\subsection*{Proof of No Common Element}
We will use proof by contradiction. Let's assume that there is at least one element, $\epsilon$ common for all the intervals.
Given that the lower bound for all the intervals is $0$, we can state that $\epsilon > 0$. Having $\epsilon$ a positive number, let's pick a natural number $N$ such that $N>\frac{1}{\epsilon}$. This means that $\epsilon>\frac{1}{N}$ which means that $\epsilon$ is not part of $A_N$ (because $\frac{1}{N}$ is the upper bound of $A_N$).\

The latest statement represents a contradiction because we assumed $\epsilon$ is common to all of the intervals and we found one ($A_N$) that $\epsilon$ doesn't belong to.

Given the proofs of inclusion and of no element commonality, we can conclude that the collection $A_n=(0, \frac{1}{n}), n \in \mathbb{N}$ satisfies the following property: $(\forall n \in \mathbb{N})((A_{n+1} \subset A_n) \land (\bigcap\limits_{n=1}^{\infty}A_n=\emptyset))$.$\square$ 

\chapter{Problem 10}
\section*{Statement}
Give an example of a family of intervals $A_n$, $n=1,2,\dots,$ such that $A_{n+1} \subset A_n$ for all $n$ and $\bigcap\limits_{n=1}^{\infty}A_n$ consists of a single real number. Prove that your example has the stated property.
\section*{Solution}
Just as a remark to start with, this problem is very similar to the problem 9. Let's consider a slight variation of the solution proposed in problem 9: we will have the collection of intervals like this: $A_n=[0, \frac{1}{n})$ where $n \in \mathbb{N}$. The proof we will follow almost the same steps as for problem 9:
\subsection*{Proof of Inclusion}
For intervals $A_n$ and $A_{n+1}$, we can state that:\noprelistbreak
\begin{enumerate}
\item they have the same lower bound, $0$.
\item because $\frac{1}{n+1}<\frac{1}{n}$, the upper bound of $A_{n+1}$ is lower that the upper bound of $A_n$.
\end{enumerate}
Given these two statements, it means that $A_{n+1} \subset A_n$.
\subsection*{Proof of Single Common Element (0)}
We will use proof by contradiction. We can state that $0$ is included in all the intervals. Let's assume that there is another element, $\epsilon \neq 0 $ common for all the intervals.
Given that the lower bound for all the intervals is $0$, we can state that $\epsilon > 0$. Having $\epsilon$ a positive number, let's pick a natural number $N$ such that $N>\frac{1}{\epsilon}$. This means that $\epsilon>\frac{1}{N}$ which means that $\epsilon$ is not part of $A_N$ (because $\frac{1}{N}$ is the upper bound of $A_N$).\

The latest statement represents a contradiction because we assumed $\epsilon$ is common to all of the intervals and we found one ($A_N$) that $\epsilon$ doesn't belong to.

Given the proofs of inclusion and of single common element, we can conclude that the collection $A_n=[0, \frac{1}{n}), n \in \mathbb{N}$ satisfies the following property: $(\forall n \in \mathbb{N})((A_{n+1} \subset A_n) \land (\bigcap\limits_{n=1}^{\infty}A_n=\{0\}))$.

As a bonus example, $A_n=(-\frac{1}{n}, \frac{1}{n})$ where $n \in \mathbb{N}$ can be a solution to this problem as well (0 being the only common element).$\square$ 

\appendix
\backmatter
%\printbibliography
%\renewcommand{\indexname}{Index}
%\printindex[document]\
\glsaddall
%\printacronyms
%\printglossary
\end{document}